\section{Notes on Design}

Smalltalk is a dynamically-typed language in the simplest sense: all
values are objects, so whether an expression is well-typed or not is
determined not by a type-checking mechanism but by the availability of
methods and message-receivers that an object has at its disposal. Thus
the expression '\$a + \$b' is ill-typed because the characters \$a and
\$b have no knowledge of the binary addition message. Because it is
possible to extend any object with any method, the interpreter defers
checking whether a statement is semantically correct until runtime
\cite{thebluebook}.

Arguments to methods are passed according to a
pass-by-object-reference strategy. Even those objects that are
primitive in other languages (e.g., integers, strings, characters) are
treated as descendants of the Object class, although there is often
some under-the-hood optimization occurring. When an object 'x' is
passed to a method as an argument, that method has access to the
actual object as if a pointer to the object was passed
\cite{postcard}. Smalltalk's runtime contains a reference counting
garbage collector. Objects that are no longer needed in a program are
deallocated as required.

Functions in Smalltalk are objects like any other value, and since all
objects are first-class, functions can be passed around as
data. Anonymous functions are supported, and are implemented as
closures. In fact, Smalltalk's semantic analyzer considers a
``block'', or sequence of actions, to comprise some combination of
lambda expressions or procedure calls that ultimately evaluates to a
single value object \cite{postcard}. While Smalltalk was envisioned as
an object-oriented language (complete with multiple inheritance,
duck-typing, and  many users refer to the language as pseudo-functional,
since it is possible to write programs using almost-exclusively
anonymous functions.

Perhaps the most interesting aspect of the Smalltalk standard library
is its support for concurrent programs. Concurrency is a property of
programs structured in such a way that it is possible to execute two
or more of its statements in parallel. Objects and subclasses
descended from Smalltalk's Process class can be spun into individual
threads by the interpreter, so program to, for example, factorize a
large number, can be partitioned into multiple processes and run
simultaneously. Execution can be halted by sending a \#suspend message
to a process object, and a process can be killed with a \#terminate
message. There are even predeclared constants that allow the user to
specify the priority of processes, which determines the relative order
of I/O and background device access \cite{postcard}.

\section{Conclusion}

Out of the hundreds of languages that see common use in industry
today, very few can lay claim to the history and impact of
Smalltalk. The almost-original object-oriented language pioneered the
use of virtual machines for interpreting, laying the groundwork for
just-in-time compilation and languages such as Java and
Python. Smalltalk introduced the world to development suites, live
debugging, and modern refactoring techniques. Should the language
someday fade into true obscurity, its legacy will continue to be felt
throughout the software development landscape. Which is, in the
opinion of the authors, all the more reason to preserve it.